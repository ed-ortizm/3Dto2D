why you think the filter needs to be defined in terms of energy rather than in terms of photons?

The flux $f_{\lambda}(\lambda)$ has units of $\displaystyle{\frac{#photos}{area\times time \times distance}}$
and $T(\lambda)$ has units of $\displaystyle{\frac{energy}{distance}}$

from Mederic: A filter can be determined in various ways. For instance what fraction of the photons is transmitted
or what fraction of the energy is transmitted. Because the photon energy depends on the wavelength, the shape of the
filter will change

fitsheader edgar.fits | grep "EXTNAME"

Con astropy.coorditates más la info del header donde tienes un pixel de referencia lo haces

f_{\lambda}(T) = \frac{\int T(\lambda) \times f_{\lambda}(\lambda)d\lambda}{\int T(\lambda)d\lambda}
